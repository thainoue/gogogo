%% v3.0 [2015/11/14]
%\documentclass[Proof,technicalreport]{ieicej}
\documentclass[technicalreport]{ieicej}
%\usepackage{graphicx}
\usepackage[T1]{fontenc}
\usepackage{lmodern}
\usepackage{textcomp}
\usepackage{latexsym}
%\usepackage[fleqn]{amsmath}
%\usepackage{amssymb}
%\usepackage[dvipdfmx]{hyperref}
\usepackage{url}

\def\IEICEJcls{\texttt{ieicej.cls}}
\def\IEICEJver{3.0}
\newcommand{\AmSLaTeX}{%
 $\mathcal A$\lower.4ex\hbox{$\!\mathcal M\!$}$\mathcal S$-\LaTeX}
%\newcommand{\PS}{{\scshape Post\-Script}}
\def\BibTeX{{\rmfamily B\kern-.05em{\scshape i\kern-.025em b}\kern-.08em
 T\kern-.1667em\lower.7ex\hbox{E}\kern-.125em X}}

\jtitle{初級学習者を対象としたアラビア語検索サイトの構築}
\etitle{Construction of an Arabic searching website for beginner-level learners}
\authorlist{%
 \authorentry[inoue.kai.n0@tufs.ac.jp]{井上 開}{Kai INOUE}{Tokyo}% 
 \authorentry[sano@tufs.ac.jp]{佐野 洋}{Hiroshi SANO}{Tokyo}% 
}
\affiliate[Tokyo]{東京外国語大学言語文化学部\hskip1zw
  〒183-8534 東京都府中市朝日町3-11-1}
 {Faculty of Engineering,
  Tokyo University of Foreign Studies\hskip1em
  Yamada 1--2--3, Minato-ku, Tokyo,
  105--0123 Japan}
\affiliate[Tokyo]{東京外国語大学大学院総合国際学研究院\hskip1zw
  〒183-8534  東京都府中市朝日町3-11-1}
 {R\&D Division, Osaka Corporation\hskip1em
  Kawada 4--5--6, Suita-shi,
  565--0456 Japan}

%\MailAddress{$\dagger$hanako@denshi.ac.jp,
% $\dagger\dagger$\{taro,jiro\}@jouhou.co.jp}

\begin{document}
\begin{jabstract}
アラビア語は国連公用語の1つであり, 全世界に3億人近い話者がいるといわれる. 日本ではまだまだ馴染みの薄いアラビア語だが, アラブ世界への社会的, 経済的関心は着実に高まってきている. これに合わせ, 日本でのアラビア語学習環境の整備を進めていくことは必須である. 

このような背景を踏まえ, 筆者はアラビア語学習教材の作成に着手した. 本稿では, 初級学習者を対象としたアラビア語検索サイト構築にいたる過程について説明する. 初級学習者を対象としたのには, 2つの理由が挙げられる. 第一に, アラビア語学習者の一人として, 教材の不足が理由で著者自身が学習初期に非常に苦労したこと. 第二に, 日本におけるアラビア語学習者の大半が初級レベルにあることである. 

検索サイトは, 電子媒体特有の機能を備えている. それは, アルファベット配列と語根配列の両立である. アラビア語の辞書は伝統的に語根配列が採用されてきたが, 初級学習者には使いこなすのが難しい. 同時に, 語根配列によるメリットも数多く存在する. 紙媒体の辞書では二つの配列を同時には採用できないというジレンマを, 電子媒体で辞書をつくることで解消している.
\end{jabstract}
\begin{jkeyword}
アラビア語
\end{jkeyword}
\begin{eabstract}
Arabic is one of the official languages of the United Nations, and it is said that there are nearly 300 million Arabic speakers all over the world.  Although Arabic is still unfamiliar here in Japan, social and economic interest to the Arabic world has been steadily increasing in recent years.  In according with this situation, it is essential to promote a development of Arabic learning environment in Japan.

Based on the above, I started to prepare an Arabic learning material.  In this paper, I will explain a process of how I got on to build an Arabic searching website for beginner-level learners.  There are two reasons why I targeted beginners.  First of all, as one of Arabic learners, my own difficulty in the beginning was due to the lack of teaching materials.  Secondly, most Arabic learners in Japan are at the beginner-level.

The searching website has a function characteristic of an electric medium.  It is a compatibility between an alphabet sequence and a root sequence.  An Arabic dictionary traditionally adopted the root sequence, but it is difficult for beginner-lever learners to use.  At the same time, there are some merits to adopt the root sequence.  By making a dictionary in an electric media, I eliminated this dilemma that we cannot adopt these two sequences at the same time.

\end{eabstract}
\begin{ekeyword}
p\LaTeXe\ class file, typesetting
\end{ekeyword}
\maketitle

\section{はじめに}
アラビア語は, 西アジアから北アフリカを中心に20以上の国および地域で話される言葉である. その形態論的特徴として, 3子音を語根としていることが挙げられる. 語根自体がそのまま語彙となることはないが, それぞれに基本的な意味が備わっている. アラビア語の語彙は, 語根に母音や接辞を付加することによって派生され, 同じ語彙から派生された語彙は意味に関連性を持っている.

KTB:書くことにまつわる語根

KaTibun:作家

一方で, 近年日本においてもアラブ世界に対しては多大なる社会的関心が寄せられてきた. 2001年の「9. 11」, 2003年からの「イラク戦争」, 2010年からは「アラブの春」, 翌年には「シリア内戦」の勃発, また「イスラム国」によるテロの脅威は現在も続いており, 2014年にはシリアのアレッポで日本人2名が拘束されるという事件があったことも記憶に新しい. またアラブ世界との経済的結びつきも無視するとはできない. 原油のほとんどを輸入に頼っている日本だが, その輸入先第1位がサウジアラビア, 第2位がアラブ首長国連邦, 両国からの輸入量を合わせると実に全体の約6割\footnote{帝国書院 \url{https://www.teikokushoin.co.jp/statistics/map/index16.html}}を占めることとなる. また日本企業の湾岸地域への進出も盛んである. 日本貿易振興機構が調査を始めた2005年から2014年の間に, アラブ首長国連邦にある日系企業拠点数は2倍以上の431事業所にまで増加している. 2020年にはドバイでの万国博覧会の開催が決定しており, 湾岸地域はこれからますます経済的盛り上がりをみせると予想される. 

このような事情を反映し, 日本でのアラビア語の学習需要も高まりつつある. 実際東京外国語大学では, 2012年にアラビア語専攻の学生の募集定員が15名から30名へと増加された. また防衛省において2011年よりアラビア語圏への留学派遣が開始されたことも、同様の事情からと推測される. 

しかし、日本でのアラビア語学習環境はいまだ整っていない。現在日本の高等教育機関でアラビア語を学習する人口のほとんどが初級レベルにあるが、初学者用の辞書は数が少なく、「パスポート 初級アラビア語辞典」が存在するのみである。しかしこの紙媒体の辞書であるために基本形からしか辞書引きできないため、語根を用いた検索による学習効果が期待できない。

一方、これらの問題を解決した電子媒体の辞書は存在するものの、初学者向けに意味の厳選されていない点や、スマートフォンでの利用に特化されていないなどの問題がある。

本研究では、東京外国語大学から出版されている「『大学のアラビア語』』単語帳」および「A Frequency Dictionary of Arabic」のデータを用いて、辞書検索サイトを構築した。これにより辞書サイトに登録する語彙を初学者に最適化させることに可能にした。

\section{アラビア語について}
アラビア語はエジプトやシリアをはじめ, 20以上の国および地域で公用語となっている. 話者総人口はおよそ3億人にものぼると言われ, 6つある国連公用語の1つにも数えられる. またアラビア語はイスラム教の原典であるコーランの言語であり, 全世界およそ16億人 のイスラム教徒が存在することを考慮に入れると, 実に世界人口の5人に1人がアラビア語と関わりを持っていることとなる. 

またアラビア語はコーランや古典文学を基盤とし, 現在では新聞やニュースなどで用いられる正則アラビア語(フスハー)と, 各地域で日常会話において用いられる口語アラビア語(アンミーヤ)に大別される. 正則アラビア語はアラブ世界での共通語として認識されているが, どこかの国および地域における母語というわけではなく, アラビア語母語話者自体も主に公教育の場で学ぶものである. つまり, 教育を受けたアラブ人は, 自身の生まれた土地で話される口語アラビア語と, 学校で学んだ正則アラビア語の両方を理解している. このよう状態は, 二重言語性と呼ばれる. 現在の日本におけるアラビア語教育は正則アラビア語が主であるため, 本論で目的とした検索サイトの構築も正則アラビア語を対象としている. 本稿で「アラビア語」は, 「正則アラビア語」のことを指す. 

\subsection{アラビア語の言語的特徴}
アラビア語は, アフロ・アジア語族のセム語派に属する屈折語である. 文字はアラビア文字を用い, 右から左に向かって記述する一方, 数字は逆に右から左に記述される. 28の子音が存在し, 中でも咽頭化ないし軟口蓋音化した子音が特徴的である. 母音は/a/, /i/, /u/の短母音とそれぞれの長母音、また/ai/, /au/などの二重母音が弁別される. 文構造はVSO型が基本だが, SVO型を取ることも多い. 名詞と形容詞は格(主格, 属格, 体格)・性(男性, 女性)・数(単数, 双数, 複数)によって変化し、定冠詞によって限定と非限定が区別される。また複数形には規則形が存在するが, 不規則変化するものが多数である. 動詞は人称(一人称, 二人称, 三人称)・性・数, および完了と非完了によって変化する. 三人称男性単数完了形は基本形と呼ばれ, 辞書の見出し語となっている.

アラビア語のほとんどの語彙は, 3つの子音から成り立っている. 3つの子音の組み合わせはそれぞれに根源的な意味を持っており, 同じ3つの子音から成る語彙は意味に関連性を持っている. この3つの子音の組み合わせのことを「語根」という. 

例えば, 「KāTiB」は「作家」を意味し, 「KiTāB」は「本」を意味する. 両単語の語根は「KTB」であり, これは「書くこと」にまつわる語根である. ここで注意したいのは, 語根はその並べ方まで含めたものであり, 語根「KTB」は, 「BKT」や「KBT」 とは異なるということだ. 厳密に言えば, 3子音の組み合わせではなく, 3子音の順列であると言える. また稀に2子音や4子音の語根も存在する. 

subsection{日本におけるアラビア語学習環境}

\section{アラビア語辞書検索システムの構築}
\subsection{基本形・語根による検索}
\subsubsection{語根からの検索}
伝統的にアラビア語の辞書では, 語根配列が採用されてきた. 上記の例で言えば, 「KāTiB」と「KiTāB」はともに辞書の「KTB」のグループにまとめられ, その見出し語として収録される. 今まで五十音順の日本語辞書, アルファベット順の英語辞書しか使ったことのない学生にとって, 語根配列のアラビア語辞書に慣れるまでにはかなりの時間がかかる. 語根配列の辞書を使用するためには, 調べたいと思っている単語の語根を判別する必要性があるからだ. この作業にはそれ相応の文法力が必要とされる. 

「maKTaBat」を辞書で引く場合, 「m」と「t」がそれぞれ語根ではなく, 派生語を作るための接辞であると見抜けなければならない. これは簡単な例であるが, 語根の中には派生された語彙において子音の一部が脱落したり, 変化したりするものがある. これらは例外として扱われるが, 例外ほど使用頻度が高いのというのが言語の常であり, アラビア語学習において骨の折れる部分である. 

「NWM」は「寝る」に関わる語根だが, これが派生した動詞「NāMa」では第二子音「W」が長母音化し, 名詞「NiyāM」では同じく第二子音「W」が「Y」に変化している. これらの例のように, 一見しただけでは語根の判別が難しい語彙も多い. こういった場合, 語根配列の辞書を引くことは格段に難しくなる. 特に文法力がまだ十分に備わっていない初級学習者は, 単語の意味を調べることさえできないという状態に陥る. これが語根配列の辞書を使用する際の大きなデメリットである. 
しかし, 逆に語根を正しく判別するだけの文法力さえあれば, 同じ語根から派生された語彙群を一覧することができるというメリットもある.  「KTB」を引けば, 「作家」, 「本」, 「図書館」という語彙を同時に知ることができる. また動詞の活用形や名詞・形容詞の複数形から辞書を引くことも可能である. 

語根配列の辞書であれば, 「KuTTāBun」が「KāTiBun」の複数形 であると知らなくとも, 語根が「KTB」であるということさえわかれば, 辞書を引いて意味を知ることできる. 「KuTuB」に関しても同様であり, これらの語彙はいずれも語根に対して母音が付加されているだけなので, 語根の判別は難しくない. それゆえ語根配列の辞書であれば, たとえこれらの語彙が初見であったとしても辞書を引くことは十分に可能である. 
以下のテンプレートに従って記述してください.
原稿執筆に際しては,本クラスファイルとともに配布される

\subsubsection{基本形からの検索}
語根配列の辞書がある一方で, 英語やフランス語などの辞書で採用されているアルファベット配列の辞書も存在する. こちらの場合, 調べたい語彙の語根を判別できる文法能力は必要とならないため, 初級学習者にとっては使い勝手がよいというメリットがある. 「NWM」のように派生語において語根の一部が, 欠落および変化してしまう語彙も, 単純にそのままの形で調べれば良いだけである. しかし一点注意しなければいけないことがある. 動詞は基本形, 名詞・形容詞は単数形が見出し語となっているため, 調べたい語彙が動詞の活用形だった場合や, 名詞・形容詞の複数形だった場合, それらを基本形ないし単数形に戻す文法力が必要であるということだ. 

上記の例であれば, 「KuTTāB」は「KāTiB」に, 「KuTuB」は「KiTāB」に戻さなければ辞書を引くことはできない. 複数形が規則変化するものである場合, 複数形から単数形を復元することはさして難しくないため, アルファベット配列の辞書ですぐに調べることができる. しかし例に挙げた名詞のように複数形が不規則変化するものの場合, 辞書を引くのは当然それらが学習者にとって初見の時であるため, 複数形から単数形を復元するのは不可能に近く, 辞書を引くことは難しくなる. このような場合には, むしろ語根辞書の方が意味を調べることに適しているといえる. 

\subsubsection{検索方法の検討}
語根配列の辞書とアルファベット配列の辞書にはそれぞれに長所と短所があり, 目的や状況に応じて使い分けることが理想的である. 紙の辞書ではどちらか一方の配列しか採用できないため, 2冊必要となってしまう. 他方, 電子媒体の辞書であれば両方を採用することができる. しかしどちらも利用できるということは, 同時にどちらか一方のみを利用し続けることを可能にしてしまう. アルファベット配列のみを利用し続けることは, 学習者の文法力養成の妨げとなる可能性も高い. アラビア語辞書において伝統的に語根配列が採用されてきたことは, 語根に対する理解がアラビア語学習において非常に重要であるからに他ならない. 

しかし榮谷\cite{sakaedani2008}も指摘するように,あまりに語根を偏重すれば, 辞書を引く能力を養うのと引き換えに,かえって語彙を覚える時間が取れないといった本末転倒な事態を招きかねない.
アルファベット配列のように, 辞書の見出し語から語彙を調べられることは初級学習者にとって利便性が高い.
本研究で構築する検索サイトの対象は初級学習者であるため, アルファベット配列を採用することに関しては問題ないだろう.
しかし, 関連する語彙が一覧できるなど語根配列によって得られるメリットも多い. 事 初級学習者であっても目的に応じて使い分けられるよう, 検索サイトでは見出し語検索と語根検索の両方を採用することとした. 同時に収録語彙数に制限を設けた. これは初級学習者が見出し語検索を過度に利用するあまり, 語根に対する理解力をいつまでも身につけられずに学習が進んで行くことを避けるためである. 初級段階の学習を終えた後は, 既存の辞書へ利用を移行して行くことを想定している. 

\subsection{アラビア語辞書データ}
2. 3. 3項で, 検索サイトの配列を検討した結果、語彙に制限を設けることとした. ここでは初学者向けの検索サイトとしてどれだけの語彙を収録すべきか検討する.

\subsubsection{語彙の大きさ}
第二言語習得においてどれだけの語彙量を目標にすべきかに関して, Nation [3]は語彙の頻度に注目することを1つの方法として挙げている. 彼は英語ネイティブスピーカーの中等学校教科書を例に, 最も頻度の高い2000語でテキストの87\%, さらにUniversity word list を加えた2800語でテキストの95\%をカバーしていると示している. この結果は英語に限られるものであり, 他言語にも同じように頻度とテキストカバー率の関係を当てはめて考えることは軽率と言わざるを得ない. 

しかしTim Buckwalter・Dilworth Parkinson [4]が, 語彙学習において頻度を1つの指標とした際, 多くの場合学習者の利益となる語彙表を得られると述べているように, 頻度が重要な指標となることは間違いない. とりわけ「初級学習者がまず学ぶべき語彙を提示する」という条件のもとでは, 非常に効果的だと考えられる.

本検索サイトでは, 見出し語として2000語を収録語彙数の指標とした. Nationが最も高頻度の語彙として2000語を区切りにしていることに加え, 東京外国語大学で利用される教科書2冊が初級学習者の幅広い利用を想定してつくられており, 合わせて約2000語を収録していることも根拠となっている.

\subsubsection{辞書データ}
辞書データは「大学のアラビア語 初級表現」, 巻末収録の単語帳の2081単語を基礎としている. 本学アラビア語専攻の教授陣が2013年度以降出版している「大学のアラビア語」シリーズは, 東京外国語大学で培われてきたアラビア語教育の集大成とも言えるものであり, 教科書としての質が高い. これらのテキストにある語彙を集めた単語帳は, 日本語の意味も入念に選別されている. これらの語彙と, 3000万語規模のコーパスから頻度上位5000語を抽出した「a frequency dictionary of Arabic」の上位2000語を照らし合わせ, 前者には含まれないが後者には含まれる818語を追加した, 全2909語を見出し語として収録している. 2081語から漏れた語彙に関しては「パスポート初級アラビア語辞典」および「Arabic English Dictionary of Modern Written Arabic」を参照した. 例文およびコロケーションに関しては, 「TUFS言語モジュール 」にアップロードされている約2000文を収録した.

\begin{thebibliography}{99}
\bibitem{washimi2016}
鷲見朗子 鷲見克典, 日本の高校と大学におけるアラビア語の教育と学習者, アラブ・イスラム研究(14), pp.103-121, 2016.

\bibitem{sakaedani2008}
榮谷温子, アラビア語辞典の語根順配列とアルファベット順配列語彙習得の観点から, 外国語教育研究(11), pp.90-100, 2008.

\bibitem{heinle1990}
I.S.P.Nation, Teaching and Learning Vocabulary, Heinle\&Heinle, 1990. 

\bibitem{buckwalter2009}
Tim Buckwalter  Parkinson Dilworth, A Frequency Dictionary of Arabic: Core Vocabulary for Learners, Routledge, 2009.

\end{thebibliography}

\end{document}
